%%%% Notes %%%%
% Indices: y_i => I is the number of target words (e), J the number of source words (f)

%todo: Check spelling of authors: Accents?
%todo: \cite{j.2018on}: Without volume, compiling breaks. But there is no volume number in the source!

\documentclass[
    a4paper,
    11pt,              % verwendete Schriftgroesse (Std: 11pt)
    DIV10,             % Bestimmt die Groesse des Textbereichs (Std: 10)
    BCOR12mm,           % Zusaetzlicher Rand auf der Innenseite (Bindekorrektur)
%    twoside,           % beidseitiger Druck
    headsepline,       % Kolumnentitel mit horizontaler Trennlinie
    cleardoublepage=empty,  % Auf Leerseiten kein Kolumnentitel und keine Seitenzahlen
    listof=totoc,       % Tabellen und Abbildungsverzeichnis ins Inhaltsverzeichnis
    bibliography=totocnumbered,  % Literaturverzeichnis ins Inhaltverzeichnis
    % idxtotoc,         % Index ins Inhaltverzeichnis
    %english,          % oder ngerman
    chapterprefix,     % Print chapter prefix before each Chapter
    %nochapterprefix,  % Don't print ``chapter'' prefix
    appendixprefix,
    numbers=noenddot,  % appendix verursacht sonst, dass ein punkt hinter die chapters gesetzt wird
    captions=tableheading,  % Tabellen Beschriftungen ueber der Tabelle
]{scrbook}             % or scrartcl - KOMA Script

\usepackage[british,ngerman]{babel}
\usepackage{natbib}

\usepackage[utf8]{inputenc}

\usepackage{graphicx}

\usepackage[svgnames]{xcolor}

\usepackage{listings}

\usepackage{adjustbox}


\usepackage{pgf}
\usepackage{pgfplots}
\pgfplotsset{compat=1.9}

\usepackage[algo2e]{algorithm2e}

\usepackage{amssymb}

\usepackage{glossaries} % for the glossary of cause 

% these packages have to be loaded _before_hyperref
% otherwise you get a "destination with the same identifier X has already been used, duplicate ignored" error 
\usepackage{float}

\usepackage[
    english,
%     a4paper,
    %bpagebackref,         % link from bibliography back to citing pages; (incompatible with biblatex)
    pdfpagelabels,
    plainpages=false,     % fuer arabische und roemische nummern wichtig
    hyperindex,
    hyperfigures,
    %linktocpage,         % Die Seitennummern als Link im Inhaltsverzeichnis anstatt des Textes
    colorlinks,           % Text als Link anstatt Box
    linkcolor=Black, %MidnightBlue,       % Linkfarbe, std=red 
    citecolor=Black, %MidnightBlue,       % cite Linkfarbe, std=green
    urlcolor=Black, %MidnightBlue,        % URL Linkfarbe, std=blue (printed version in black, online version in blue)
    breaklinks,           % Links ueber mehrere Zeilen moeglich
    bookmarks,            % Acrobat Bookmarks erstellen
    bookmarksopen=true,   % Acrobat Bookmarks beim oeffnen des Dokumentes anzeigen
    bookmarksnumbered=true,
    pdfproducer={LaTeX with hyperref and thumbpdf},
    pdfstartpage={1},     % Startseite 
    %pdfstartview={Fit},
    %pdfview={FitH},
    %pdffitwindow=true    
]{hyperref}


\usepackage{algorithm}
\usepackage[noend]{algpseudocode}
\usepackage{algorithmicx,algpseudocode}
  \makeatletter
  \def\theHALC@line{\thealgorithm-\theALC@line}
  \def\theHALC@rem{\thealgorithm-\theALC@rem}
  \makeatother

\usepackage{needspace} %To prevent line break where I don't want it
\usepackage{fix-cm} % to build huge font size


%\usepackage{cite} %Allows citations to be broken at the end of a line

\usepackage{enumerate} %enumerate formating

\usepackage[caption=false]{subfig} %subfloat
%\usepackage{subfigure}
%\addtolength{\subfigcapskip}{-0.2in}


\usepackage{longtable}

%Chinese
%\usepackage{CJK}
\newcommand{\cninput}[1]{\hspace{-0.3cm}\begin{CJK}{UTF8}{gbsn}#1\end{CJK}}

\usepackage{tikz}
\usetikzlibrary{arrows,positioning} 
\usetikzlibrary{shapes,decorations}
\usetikzlibrary{calc}
\usetikzlibrary{decorations.pathreplacing}

\tikzset{
    %Define standard arrow tip
    >=stealth',
    %Define style for boxes
    punkt/.style={
           rectangle,
           rounded corners,
           draw=black, very thick,
           text width=6.5em,
           minimum height=2em,
           text centered},
    % Define arrow style
    pil/.style={
           ->,
           thick,
           shorten <=2pt,
           shorten >=2pt,}
}
\tikzstyle{word} = [draw, top color=blue!10, bottom color=blue!50, rounded corners=1mm]
\tikzstyle{wordRed} = [draw, top color=red!10, bottom color=red!90, rounded corners=1mm]
\tikzstyle{wordGreen} = [draw, top color=green!10, bottom color=green!90, rounded corners=1mm]
\tikzstyle{wordOrange} = [draw, top color=orange!10, bottom color=orange!90, rounded corners=1mm]

\usepackage{amsmath,bm} %bm avoids horizontal whitespace between bolded characters and index

%used for rp helper command
\usepackage{fp}
\usepackage{numprint}

\newcommand{\Sig}[1]{\textcolor{blue}{#1}} % significant to 95%
\newcommand{\sig}[1]{\textcolor{magenta}{#1}} % significant to 90%
\newcommand{\significanceExplanation}{Significant improvements over the baseline}

\newcommand{\cev}[1]{\rule{0mm}{0mm}\,\,\mathchoice%
{\reflectbox{$\displaystyle\vec{\reflectbox{$\displaystyle\!\!#1$}}$}}%
{\reflectbox{$\vec{\reflectbox{$\!\!#1$}}$}}%
{\reflectbox{$\scriptstyle\vec{\reflectbox{$\scriptstyle\!\!#1$}}$}}%
{\reflectbox{$\scriptscriptstyle\vec{\reflectbox{$\scriptscriptstyle\!\!#1$}}$}}%
}

\newcommand{\significanceKey}{
\begin{flushright}
\hspace{-1cm}\begin{tabular}{lll}
  \sig{$p < .1$}  & significance \\
  \Sig{$p < .05$} & significance \\
\end{tabular}
\end{flushright}
}

% Hurenkinder und Schusterjungen verhindern
\clubpenalty10000
\widowpenalty10000
\displaywidowpenalty=10000

%%%%%%%%%%%%%%%%%%%%%%%%%
% own helper functions
%%%%%%%%%%%%%%%%%%%%%%%%%
%Rounds numbers to percent
\newcommand{\rp}[1]{%
  \FPset{\per}{#1}%
  \FPmul{\per}{\per}{100}%
  \FPround{\per}{\per}{1}%
  \numprint{\per}%
}%

\newcommand{\percent}[2]{%
  \FPset{\per}{#1}%
  \FPdiv{\per}{#1}{#2}
  \rp{\per}\%
}%



\usepackage{theorem}                            % instead of \usepackage{amsthm}
\theoremstyle{break}
\theorembodyfont{\itshape}	
\theoremheaderfont{\scshape}
\newtheorem{Cor}{Corollary}
\newtheorem{Def}{Definition}
\newtheorem{Rmk}{Remark}
\newtheorem{Theorem}{Theorem}
\newtheorem{Notation}{Notation}

\newcommand\T{\rule{0pt}{2.2ex}}       % Top strut

% @ environment %%%%%%%%%%%%%%%%%%%%%%%%%%%%%%%%%%%%%%%%%%%%%%%%%%%%%%%%%%%%%%%%
\usepackage{xspace}                             % context sensitive space after macros
\makeatletter 
\DeclareRobustCommand\onedot{\futurelet\@let@token\@onedot}
\def\@onedot{\ifx\@let@token.\else.\null\fi\xspace}
\def\eg{{e.g}\onedot} \def\Eg{{E.g}\onedot}
\def\ie{{i.e}\onedot} \def\Ie{{I.e}\onedot}
\def\cf{{c.f}\onedot} \def\Cf{{C.f}\onedot}
\def\etc{{etc}\onedot} \def\vs{{vs}\onedot} 
\def\wrt{w.r.t\onedot} \def\dof{d.o.f\onedot}
\def\etal{{et al}\onedot}
\def\zB{z.B\onedot} \def\ZB{Z.B\onedot}
\def\dh{d.h\onedot} \def\Dh{D.h\onedot}
% %%%%%%%%%%%%%%%%%%%%%%%%%%%%%%%%%%%%%%%%%%%%%%%%%%%%%%%%%%%%%%%%%%%%%%%%%%%%%%%

\usepackage{mfirstuc} % capitalize words using \capitalisewords

%TikZ code "illustrating" the new "brace"
\newcommand{\newbrace}[1][]{
\begin{tikzpicture}[baseline=-0.5ex]
\draw[#1] (0,0) -- (0.3,0.3);
\draw[#1] (0,0) -- (0.3,-0.3);
\end{tikzpicture}
}

% the optional argument allows you to select the type of arrow 
% you can also customize the "new brace"
\newenvironment{casesnew}[1][->]%
{\;\newbrace[#1]\;\begin{array}{@{}l@{}}}%
{\end{array}}

\newcommand{\publicationDate}{27. M\"arz 2018}
\subject{Masterarbeit im Fach Informatik}
\title{Unsupervised Learning of Neural Network Lexicon and
	Cross-lingual Word Embedding}
\newcommand{\Surname}{LastName}
\newcommand{\Firstname}{Firstname}
\newcommand{\Aachen}{Aachen}
\newcommand{\studentname}{\Firstname~\Surname}
\newcommand{\MatrikelNummer}{Number}
\author{\studentname}


%Global commands
%\newcommand{\argmax}{\operatornamewithlimits{arg\,max}}
%\DeclareMathOperator*{\argmax}{arg\,max} % I'm not sure where the difference is between this one and operatornamewithlimits
%\DeclareMathOperator*{\argmin}{arg\,min} % I'm not sure where the difference is between this one and operatornamewithlimits

%\newcommand{\argmax}{\operatornamewithlimits{argmax}}
%\newcommand{\argmin}{\operatornamewithlimits{argmin}}
\newcommand{\argmax}[1]{\underset{#1}{\operatorname{argmax}}}
\newcommand{\argmin}[1]{\underset{#1}{\operatorname{argmin}}}


\newcommand{\phraseset}[1]{\ensuremath{\mathcal{#1}}}
\newcommand\BLEU{\textsc{Bleu}}
\newcommand\TER{\textsc{Ter}}
\newcommand{\PPL}{\textsc{Ppl}}

\newcommand{\GIZA}{{GIZA\nolinebreak[4]\hspace{-.025em}\raisebox{.2ex}{\small\bf++}}\xspace}

\newcommand{\leftphrasedelim}{\mathlarger{\langle}}
\newcommand{\rightphrasedelim}{\mathlarger{\rangle}}
\newcommand{\kstar}{^\star}
\newcommand{\ntind}[1][]{^{\sim#1}} % ``Non-terminal'' index

\newcommand{\phrasepair}[2]{\ensuremath{\leftphrasedelim}#1, #2\ensuremath{\rightphrasedelim}}
\newcommand{\rulearrow}{\rightarrow}
\newcommand{\gap}{\ \ensuremath{\Diamond}\ }

% Indentation
\newcommand{\ind}{\hspace*{5mm}}

%fix figure placement
\renewcommand{\topfraction}{0.85}
\renewcommand{\textfraction}{0.1}
\renewcommand{\floatpagefraction}{0.75}


%\input{figures}
\makeglossaries

% \usepackage{showframe}

%%%%%%%%%%%%% custom packages %%%%%%%%%%%%%%
\patchcmd{\@setref}{\bfseries ??}{\bfseries\color{red} undefined Label}{}{} %makes broken references (by \ref) more visible
\usepackage{forest}
\usepackage{textcomp}
\usepackage{booktabs} %nice tables
\usepackage{needspace} %to avoid page break to soon after section header

\usepackage{tablefootnote}


% https://tex.stackexchange.com/a/88963/118090
% bold with same width - allows to align numbers in tables
\newcommand{\bftab}{\fontseries{b}\selectfont}

\usepackage{mathtools} % allow centered equal signs with text above them
\newlength{\leftstackrelawd}
\newlength{\leftstackrelbwd}
\def\leftstackrel#1#2{\settowidth{\leftstackrelawd}%
	{${{}^{#1}}$}\settowidth{\leftstackrelbwd}{$#2$}%
	\addtolength{\leftstackrelawd}{-\leftstackrelbwd}%
	\leavevmode\ifthenelse{\lengthtest{\leftstackrelawd>0pt}}%
	{\kern-.5\leftstackrelawd}{}\mathrel{\mathop{#2}\limits^{#1}}}
\newcommand{\eqtext}[1]{\leftstackrel{\mathrm{#1}}{=}}
\newcommand{\equivtext}[1]{\leftstackrel{\mathrm{#1}}{\Leftrightarrow}}

\usepackage[nameinlink,capitalize,noabbrev]{cleveref} % references \cref and \Cref
\creflabelformat{equation}{#2\textup{#1}#3} %no paranthesis for equations
\DeclareRobustCommand{\abbrevcrefs}{% \cshref uses abbreviations, used within equations (see eg. derivations)
	\crefname{figure}{fig.}{figs.}%
	\crefname{equation}{Eq.}{Eqs.}%
}
\DeclareRobustCommand{\cshref}[1]{{\abbrevcrefs\cref{#1}}}

% https://tex.stackexchange.com/a/65130/118090
% define "struts", as suggested by Claudio Beccari in
%    a piece in TeX and TUG News, Vol. 2, 1993.
\newcommand\Tstrut{\rule{0pt}{2.6ex}}         % = `top' strut
\newcommand\Bstrut{\rule[-0.9ex]{0pt}{0pt}}   % = `bottom' strut



\babelhyphenation[british]{peep-holes}
\babelhyphenation[british]{ex-pec-ta-tion}
\babelhyphenation[british]{mea-sured}
\babelhyphenation[british]{mathe-matcis}
\babelhyphenation[british]{RETURNN}
\babelhyphenation[british]{in-for-ma-tion}

\usepackage{tkz-fct}

\usetikzlibrary{positioning, fit, arrows.meta}

\tikzstyle{neuronSmall}=[rectangle,
thick,
line width=0.5mm,
minimum size=1.2cm,
draw=black,
text centered,
minimum height=3em,
fill=white!20]



%taken from att_nmt.tex for the figure of attention
% define the shape of nodes here!
\tikzstyle{neuron}=[circle,
thick,
line width=0.5mm,
minimum size=1.3cm,
draw=black,
text centered,
minimum height=2em,
fill=white!20]
\tikzstyle{neuronSmall}=[circle,
thick,
line width=0.5mm,
minimum size=0.7cm,
draw=black,
text centered,
minimum height=2em,
fill=white!20]
\tikzstyle{neuronNoBorder}=[circle,
thick,
minimum size=1.2cm,
draw=none,
fill=white!20]   



% glossary entries
\newacronym{NMT}{NMT}{neural machine translation}


\begin{document}

\cleardoublepage
\graphicspath{{pictures/}}
\include{glossary}
\frontmatter

\pdfbookmark[0]{Title}{front}

% Festes Datum
\date{\publicationDate}
\thispagestyle{empty}
\begin{titlepage}
  \begin{minipage}[h]{14.5cm}
  \parindent=0pt
  \raggedleft
 
% \includegraphics[width=12mm]{logo_i6_small_blue}\\
\vspace{0.4cm}

  \Large
  Masterarbeit im Fach Informatik\\
  \textsc{Rheinisch-Westf\"alische~Technische~Hochschule~Aachen}\\
  Lehrstuhl f\"ur Informatik 6\\
  Prof. Dr.-Ing. H. Ney\\%[0.7cm]
 %\rule{0cm}{0cm}
  \vspace{0.4cm}
  \rule{\textwidth}{1mm}
  %\sf
  \makeatletter \Huge \center
  \textbf{\@title}\\
  \rule{\textwidth}{1mm}
  \raggedleft
  \vspace{2cm}
  \Large

  \@date\\
  \vspace{0.4cm}

  vorgelegt von:
  \\ Autor Jiahui Geng
  \\ Matrikelnummer 365655
  %\\ E-Mail \href{mailto:felix.rietig@rwth-aachen.de}{felix.rietig@rwth-aachen.de}
  \\[2eX]
  Gutachter:\\
  Prof.~Dr.-Ing.~H.~Ney\\
  Prof.~B.~Leibe,~Ph.~D.\\[2eX]
  Betreuer:\\
  M.Sc.\ Yunsu Kim
  \makeatother
  \vfill\vfill
  \end{minipage}
\end{titlepage}


\selectlanguage{ngerman}
{
\renewcommand*\chapterheadstartvskip{\vspace*{-5\topskip}} % the assurance needs more space, so move the title up with this
% \chapter*{Erkl\"arung}
%%\vspace{10cm}
%Hiermit versichere ich, dass ich die vorliegende Masterarbeit
%selbstst\"andig verfasst und keine anderen als die angegebenen Quellen und Hilfsmittel
%verwendet habe. Alle Textausz\"uge und Grafiken, die sinngem\"a\ss\ oder
%w\"ortlich aus ver\"offentlichten Schriften entnommen wurden, sind durch
%Re\-fe\-ren\-zen ge\-kenn\-zeichnet.%\cite{peitz:2010} \\[3ex]
% 
% Hiermit versichere ich, diese Arbeit selbstständig verfasst zu haben und keine anderen als die angegebenen Quellen und Hilfsmittel benutzt sowie Zitate kenntlich gemacht zu haben. \\[3ex]
% 
% Aachen, \publicationDate\\[7ex]
% 
% Arne Nix

%%% Local Variables: 
%%% mode: latex
%%% TeX-master: "da"
%%% End: 
% \@openrighttrue


\chapter*{Erkl\"arung}
\vfill
% \vspace{2cm}
% \pagenumbering{gobble}
% \begin{center}
\begin{tikzpicture}[overlay, text height=1.5ex, text depth=0.25ex, yshift=0.5mm]
\node[anchor=south west,inner sep=0] at (0,0)  {\includegraphics[width=0.95\linewidth, trim={2.5cm 3cm 2.5cm 4cm}, clip]{Formular_Eidesstattliche_Versicherung}};
\node [anchor=south west, inner sep=0] at (0cm, 18.4cm) {\Large \Surname, \Firstname};
\node [anchor=south west, inner sep=0] at (7.2cm, 18.4cm) {\Large \MatrikelNummer};
\node [anchor=south west, inner sep=0] at (0cm, 15.9cm) {\large \makeatletter \@title \makeatother};
\node [anchor=south west, inner sep=0] at (0cm, 11.2cm) {\large \Aachen, \publicationDate};
\node [anchor=south west, inner sep=0] at (0cm, 0.5cm) {\large \Aachen, \publicationDate};
\draw [line cap=round, line width=0.5mm] (8.8cm, 17.11cm) -- (9.65cm, 17.11cm); % cross out "Arbeit"
%\draw [line cap=round, line width=0.5mm] (9.75cm, 17.11cm) -- (11.70cm, 17.11cm); % cross out "Arbeit"
\draw [line cap=round, line width=0.5mm] (0cm, 16.7cm) -- (1.8cm, 16.7cm); % cross out "Masterarbeit"
\end{tikzpicture}
% \end{center}


% \includegraphics[width=0.95\linewidth, trim={2.5cm 2.9cm 2.5cm 3.9cm}, clip]{Formular_Eidesstattliche_Versicherung_neu}

} 

\selectlanguage{british}
\chapter{Abstract}
neural machine translation systems(NMT) beyond traditional statistical machine translation(SMT) in

This thesis start from unsupervised word translation,  

%\input{acknowledgements}

\pdfbookmark[0]{Contents}{toc} % ziffer fuer ebene
\glsunsetall
\tableofcontents

\mainmatter          % Hauptteil (arabische Seitenzahlen)

\glsresetall

\chapter{Introduction}




\section{Neural Network Lexicon}


\section{Cross-lingual Word Embedding}

%\cite{mikolov2013distributed}
%\cite{artetxe2017unsupervised}


\glsresetall
\chapter{Machine Translation}
\section{Rule-based machine translation}


Rule-based machine translation is machine translation system based on linguistic information. An RBMT system generates translation based on different levels of analysis, e.g. part of speech tagging (POS), morphological analysis,  semantic analysis, constituent analysis, dependency analysis.

Rule-based machine translation has the following advantages: No bilingual texts are required. Also because RBMT are built on a source language analysis and the target language generator and the source analysis part and target generation part are seperate, so it can be shared between multiple translation system if we replace the part with an other part for a closed related language.
However the method is still not so general, we need to build the dictionary and  linguistic rule set manually, it is very expensive. also it is hard to deal with rule interactions in big systems, ambiguity and idiomatic expression.




\section{Statistical machine translation}
The initial models for machine translation are based on words as units (Word-based machine translation), that can be translated, inserted, dropped and reordered.
Fertility is the notion that input words produce a specific number of output words in the output language.

Define the phrase-based statistical machine translation model mathematically.  First apply the Bayes rule to invert the translation direction and integrate a language model $p_{LM}$ so the best English translation for the input sentence ${f} $ is defined as 
Reordering is handled by a distance-based reordering model.
\section{Neural machine translation}


\glsresetall
\chapter{Word Embedding}
\cite{xing2015normalized}
\section{Monolingual Embedding}
\subsection{CBOW and Skip-gram Model}
	Word embeddings is distributed representation of words in a vector space. With the learning algorithm it can capture the contextual or co-occurrence information. The word embedding has an interesting and important property: similar words will have similar distribution in the embedding space, with that property, we can find meaningful near-synonyms or  Some successful methods for learning word embedding like word2vec  \cite{mikolov2013distributed}
	Continuous Bag-of-Words model(CBOW) and Skip-Gram model
	CBOW model and Skip-Gram model are currently the  common structures to learn the word embedding. Algorithmically,  CBOW tries to predict the current word based on the context while Skip-Gram model tries to maximize classification of a word based on another word in the same sentence.
	The neural probability language model defines the prediction probability using the softmax function:

	
	\begin{align}
	p(w_t | w_s) & = \textrm{softmax} {(s(w_t, w_s))} \\
	& = \frac{\exp\{s(w_t, w_s)\}}{\sum_{w^{\prime} \in W}{\exp\{s( w^{\prime}, w_s)\}}} 
	\end{align}
	where ${w_t}$ is target word)label word), ${w_s}$ is the source word(input word), for Skip-Gram model, the target word refers to the context words, the source word refers to the current word, for CBOW model is simply inverted. ${W}$ denotes the whole vocabulary. Then the training objective of the model is to maximize the log-likelihood on the training dataset, i.e. by maximizing:
	
	\begin{align}
	J_{ML} & = \log p(w_t| w_s)	\\
	& = s(w_t, w_s) - \log(\sum_{w^\prime \in W} {\exp\{s(w^\prime, w_s)\}})	
	\end{align}


	
	However the normalization on the whole vocabulary is very expensive because it is conducted for all words at every training step. The problem of predicting words can be considered as an independent binary classification task. For example in the Skip-Gram model, we consider all the context words as positive samples and the words randomly sampled from the dictionary as the negative ones. Then the training objective is 
	\[J_{NEG} = \log {Q_{\theta}{(D=1 | w^{\prime}, w_s)}} + \sum_{w^{\prime} \sim W} {\log{Q_{\theta}{(D=0 | w^{\prime}, w_s )}}}  \]
	
	where ${Q_{\theta}{(D=1| w^{\prime} w_s)}}$ is the binary logistic regression probability. In practice, we draw k contrastive words from the noise distribution. Since we only calculate the loss function for k samples instead the whole vocabulary, it becomes much faster to train.
	
	
	\[\frac{1}{T} \sum_{t=1}^{T} \sum_{-c<j<c, j\neq 0}{\textrm{log}{p(w_{t+j}|w_t)}}\]
	where c is the size of training context, larger context size make the results more precise at the cost of training time. Suppose we are give a scoring function to evaluate the word pair(word, context), the Skip-Gram model
	
	\[\frac{1}{T} \sum_{t=1}^{T} \sum_{-c<j<c, j\neq 0}{\textrm{log}{p(w_{t}|w_{t+j})}}\]
	 According to  empirical results, CBOW works better on smaller datasets because CBOW smoothes over a lot of the distributional information while Skip-Gram model performs better when we have larger datasets
	
	
	Noise-Contrastive Training
	
	
	\subsection{fastText}
	The training methods above treat each word as a distinct word embedding, however intuitively we can obtain more information from the morphological information of words. A subword model was proposed to try to fix such problem.The training network is similar, the model design a new presentation of the word: it adds speicial symbols $<$, ${>}$ as boundary information at the beginning and the end of a word. Then a normal word is represented as a bag of character $n$-grams . For example the word "where" and n equals 3, the it can be represented as the following 5 tri-grams: 
	\[ <wh, whe, her, ere, re>\]
	Suppose in this way we denote a word ${w}$ as ${G_{w}}$ the set of character ${n}$-grams, we assign for each character ${n}$-gram $g$ in ${G_{w}}$, we assign a distinct vector $z_g$, we will finally represent the embedding of word ${w}$ as the sum of these vector and also for the scoring function:
	\[s(w, w_s) = \sum_{g \in G_{w}} z_g^{T} w_s \]
	
\section{Supervised Learning of Cross-lingual Word Embedding}



	Cross-lingual word embedding is defined as word embedding of multiple languages in a joint embedding space. Mikolov first notice that the embedding distributions exhibit similar structure across languages. They proposed to use a linear mapping from the source embedding to target embedding. \\
	\begin{figure}[t]
		\includegraphics[width=14cm]{crossembedding}
		\centering
		\caption{A cross-lingual embedding space between German and English (\cite{ruder2017survey})}
	\end{figure}
	
	

	In the thesis, I assume there are two set of embeddings ${e}$, ${f}$trained separately on monolingual data.  The propose of cross-lingual word embedding training is to learn such a mapping ${W \in }$ from source embedding space to target embedding space, so $Wf_i, e_j$ in the same embedding space and for all corresponding word pairs, we need to optimize the mapping ${W}$, so that" 
	\[ \arg\min_{W \in R^{d \times d}} \sum_{i} \lVert Wf_i - e_i \rVert \]
	where $d$ is the dimension of embeddings, and the distance ${\lVert Wf_i - e_i \rVert}$ can be different types. We prefer the Euclidean distance.  

first observe that word embeddings trained separately on monolingual corpora exhibits isomorphic structure across languages, as illustrated in Figure {}. That means we can create a connection between source embedding and target embedding even with simple linear mapping. This has far-reaching implication on low-resource scenarios {}{}{}, because word embedding requires only plain text to train, which is the most abundant form of linguistic resource.
	

\subsection{***}
According to the training method we can divide the supervised method into three:
\begin{enumerate}
	\item Mapping based approaches\\
	First train the monolingual word embedding separately and then seek the seed dictionary to learn the mapping. 
	\item Pseudo-multi-lingual corpora-based approaches\\
	Use the monolingual embedding training method on constructed corpora that contains both the source and the target language.
	\item Joint methods\\
	Take the parallel text as input and minimize the source and target language losses jointly with the cross-lingual regularization term
\end{enumerate}

A dictionary is necessary for learning the cross-lingual word embedding. 
minimizing the distance in a bilingual dictionary.\\

Xing \cite{ } showed that the results are improved when we constrain the ${W}$ to be an orthogonal matrix. This constraint,  the optimal transformation can be efficiently calculated in linear time with respect to the vocabulary size.

The problem then is simplified as the Procrustes problem and there exists a closed-form solution obtained from the SVD of ${EF^T}$

\subsection{Orthogonal Constraints}
Starting from \cite{mikolov2013exploiting}, the mapping from source embedding space to target embedding space can be represented as a linear repression. The objective can be defined as:
\[ \min_{W} \sum_{i} {\lvert Wf - e  \rVert}^2 \]
Since we retrieve the word translation according to cosine similarity, it's better to solve the problem by redefine the optimization function using the cosine distance:
\[ \argmax{W} {\sum_{i} (W f_i)^T e^i} \]. We consider the source and target embedding in the same space. In this case, the normalization constraint on word vectors can be satisfied by constraining $W$ as an orthogonal matrix. 
This is equivalent to minimizing the (squared) Frobenius norm of the residual matrix:
\[ W^* = \argmin{W} {\lVert WF - E \rVert}^2_F \]
The problem boils down to the Procrustes problem which has a closed form solution obtained from the singular value decomposition (SVD).
\[W* = = UV^T , \quad U\Sigma V^T = SVD(EF^T) \]
\subsection{CSLS Loss}
Inspired from the work of \cite{conneau2017word}, where the dictionary inducted from CSLS loss: ${\bm{e}}$
\[ CSLS(\bm{e} ,\bm{f}) = -2 cos(\bm{e}, \bm{f}) + \frac{1}{k} \sum_{\bm{e}^{\prime} \in N_{\bm{e}}(\bm{f})} {cos(\bm{e}^{\prime}, \bm{f})}+ \frac{1}{k}  \sum_{\bm{f}^{\prime} \in N_{\bm{f}}(\bm{e})   } {cos(\bm{f}^{\prime}, \bm{e})}\]
since we have ${cos(\bm{We}, \bm{f}) = \bm{e}^T \bm{W}^T \bm{f}}$

The loss function can be rewritten as:
\[ \min_{\bm{W} \in } = \frac{1}{n} \sum_{i=1}^{n} \]


Minimization of a non-smooth cost function over the manifold of orthogonal matrices . Instead of using manifold optimization tools, \cite{bibid} proposed to derive convex relaxations that can lead to a simple and tractable minimization algorithm.
\begin{itemize}
	\item Spectral norm:
	replacing the set of orthogonal matrices $1$ by its convex hull, that is the set of matrices with singular values smaller than 1, the unit ball of the spectral norm 
	\item Frobenius norm:
	replacing the the set of orthogonal matrices $1$ with ball of radium $\sqrt{d}$ in Frobenius norm
\end{itemize}
 
With such two relaxations, the CSLS loss is constrained to a convex function with respect to the mapping $\bm{W}$.

We train the linear mapping with in the spectral norm by projected gradient descent:$2$ 
For each iteration, train the mapping $\bm{W}$ with gradient descent, then constrain the mapping by projection of the set.
\begin{itemize}
	\item Spectral norm\\
	take the SVD of the trained matrix, threshold the singular values to one
	\item Probenius norm\\
	divide the matrix by its Frobenius norm
\end{itemize} 



	
	

\glsresetall
\chapter{Sentence Translation}
As mentioned previously, training a traditional machine translation system requires large parallel data. With cross-lingual word embedding we can already find ambiguous word translations, in this chapter I propose a simple yet effective method to improve quality of translation which starts from the word-by-word translation. We integrate additional models, such as language model of the target side and denoising neural network of the target side to produce meaningful sentence translation. Since all the our models  are trained on monolingual corpora, This  method is fully unsupervised. Such system surpasses state-of-the-art unsupervised NMT without costly iteratively training.  
\section{Context-aware Beam Search}
	\subsection{Language Model}
		Language models are widely applied in NLP tasks, they assign a probability to a sequence of words so that they can improve the quality of systems outputs like machine translation, spell correction, speech recognition and question-answering, etc.,.According to the similarity of cross-lingual word embedding, we are able to find some meaningful for translation candidates for a given word. But there are also words that actually noise in the candidates or obviously incorrect because of grammar checking. With the support of language model, we can select the most probable words given previous word translation candidates. 
		
		${N}$-gram language models use the Markov assumption to break the probability of a sentence into the product of the probability of each word given a limit history of preceding words. 
		\[ p(e_1^I) = \prod_{i=1}^{I} p(e_i| e_1, \cdots	e_{i-1}) = \prod_{i=1}^I {p(e_i | e_{i-(I-1)}, \cdots , e_{i-1})}  \] 
		
		The conditional probability can be calculated from ${N}$-gram model frequent counts:
		\[p(e_i | e_{i-(n-1)}, \cdots , e_{i-1}) = \frac{count(e_{i-(n-1)}, \cdots, e_i)}{count(e_{i-(n-1)}, \cdots, e_{i-1})} \]
		Language model can handle sparse data problem. Some words or phrases have not been seen yet in the training corpus does not mean they are not impossible. Different smoothing techniques like back-off or interpolation are implemented to assign a probability mass to unseen cases.
	\subsection{Beam Search}
	When using LM, the complexity of a search graph is exponential to the length of the given source sentence. Beam search is a heuristic search algorithm that explores a graph by expanding the most promising nodes. At each step of the search process, it will evaluate all the candidates together with the reserved translation results from last step, it will only store a predetermined number (beam size) of translations for next step. The greater the beam size is, the fewer states will be pruned. 	
	Tt is suggested to prune poor word translation candidates as soon as possible to reduce the search space and speed up the translation. 
	
	In this thesis, translation system does not handle reordering, it combines the language model and lexicon model to construct a word-by-word framework, where the output length equals that of input sentence. Since the lexicon model actually gives a cosine similarity instead of a normalized probability, we introduce weights $\lambda_{LM}$ and $\lambda_{lex}$ to scale the contribution of each of the two components:
	\[ \hat{e}_1^J = \argmax{e_1^J}{\ \prod_{i=1}^{J}} {p^{\lambda_{LM}}(e_i|e_{i-(n-1)}^{i-1}) \cdot q^{\lambda_{lex}}(f_i,e_i)}\]

 	where the lexicon score ${q(f,e) \in [0,1]}$ defined as:
 	\[q(f,e) = \frac{d(f,e)+1}{2} \]
 	${d(f,e)\in [-1,1]}$ cosine similarity between ${f}$ and ${e}$
	
	
	In experiments, we find linear scaling works better than others, e.g. sigmoid or softmax.
	
\section{Denoising Neural Network}
\subsection{Denoising Auto-encoder (DAE)}
	With the help of language model, we improve the quality of word-by-word translation but the results are still far from acceptable. Because of the natural defect of word-by-word translation, word sequence in translation keeps the same as in input. This is contrary to our knowledge that different languages have different grammars also sequences. We implement the sequential DAE to improve the translation output.
	
%	An autoencoder is a neural network that is trained to copy its input to the output. Autoencoders minimize the loss function like: 
%	\[ L(\bm x, g(f(\bm x))) \]
	Autoencoder is a type of neural network model used to learn efficient data coding, typically dimension reduction in an unsupervised manner. The DAE is to force the hidden layer to discover more robust features by training the autoencoder to reconstruct the input from a corrupted version. It does two things, try to encode the input and try to undo the effect of a corrupt process stochastically applied to the input. In our model, the corrupt input is the word-by-word translation and the output ought to be the standard translation with correct sequence. Since we do not have parallel data as the references. We need to model the denoising process with artificial parallel data. We design three types of noises including a novel noise type to mimic the corrupted sentence. The loss function for DAE is defined as: 
	\[ \mathcal{L}^{auto} = \mathbb{E}_{e_1^I \sim \mathcal{E}}[-\log P(e_1^I| \text{noise}(e_1^I))] \]
	where $\text{noise}(e_1^I)$ is noisy target sentences where artificial noises are added.

	
\subsection{Noise Model}
We design three types of noise to handle the fertility and reordering problem, namely reordering noise, insertion noise and deletion noise. 

%In experiments, the noise model can improve the sentence translation, but since it actually starts from the word-by-word translation, it can only deal with reordering in limited distance, cannot work for global reordering.\\

	\begin{figure}[h]
	\includegraphics[width=14cm]{denoising}
	\caption{ Reordering noise}
	\centering
\end{figure}
	\textbf{Reordering Noise}\\

	The reordering problem is a common phenomenon in the word-by-word translation since the word order in source language is not the same in target language. 
	For example, in  the grammar of German, the verb is often placed at the end of the clause. 
	``um etwas zu tun". However in English, it is not the case; the corresponding translation sequence is ``to do something". The verb should always before the noun.
	In our beam search, LM only assists in choosing  more suitable word from the translation candidates, and cannot reorder the word sequence at all.
	
	For a clean sentence from the target monolingual corpora, we corrupt the word sequence by permutation operation. We limit the maximum distance between the original position and its new position.
	
	The design of reordering noise is as followed:
	\begin{enumerate}
		\item For each position ${i}$, sample an integer ${\delta_i}$ from ${[0, d_{per}]}$
		\item Add ${\delta_{i}}$ to index ${i}$ and sort ${i+\delta_{i}}$
		\item Rearrange the words to be in the new positions, to which where indices have been moved
	\end{enumerate}

	Reordering actually depends on specific language pair. The reordering noise here just models the most general case.
%	However in the experiments we found the performance of the denoising network aimed at such noise is not obvious. The Bleu score before and after the process is close.\\

	
	\textbf{Insertion Noise}\\
	\begin{figure}[ht]
	\includegraphics[width=14cm]{insertion}
	\caption{Insertion noise}
	\centering
\end{figure}	

	The word-by-word translation system predict a target word at every source position of the sentence. However,  the vocabularies of different languages are not symmetric. For example, in German, there are more compound words than that in English. So when translating between languages, there are a plenty of cases that a single word will be translated to multiple words and multiple words correspond to a single word conversely. For example: from a German sentence: ``ich höre zu" to ``i'm listening". A very frequent word ``zu" which corresponds to ``to" in English, is dropped from the sentence. The design of reordering noise is as followed:
	\begin{enumerate}	
		\item For each position ${i}$, sample a probability ${p_i \sim \textrm{Uniform}(0,1)}$
		\item If ${p_i} < p_{ins}$, sample a word ${e}$ from the most frequent ${V_{ins}}$ target words and insert it before the position${i}$
	\end{enumerate}

	We limit the insertion word in a set consisting of the top frequent word in the target language ${V_{ins}}$ \\
	
	




	\textbf{Deletion Noise}\\
	The deletion noise is just a contrary case of insertion noise.
	Because we are limited to generate only one word per source word, it is also possible that a target word in the reference is not related to any source word.  For example for ``eine der besten" the corresponding translation is ``one of the best". We need to add an extra preposition in the target sentence.  To simulate such situation, we drop some words randomly from a clean target sentence.
	
	\begin{enumerate}
		\item For each position i, sample a probability ${p_i \sim \textrm{Uniform}(0,1)}$
		\item If ${p_i} < p_{del}$, drop the word in the position i
	\end{enumerate}
	
		\begin{figure}[h]
		\includegraphics[width=14cm]{deletion}
		\caption{ Deletion noise}
		\centering
	\end{figure}

	
	
	
	
	
	
	
	
	
	
	
	
	
	
	
	
	
	
	
	
	
	
	
	
	
	
	
	


\appendix
\input{appendix}



\backmatter          % Nachspann einleiten (wie frontmatter)
\listoffigures
\listoftables

% \nocite{*}

\printglossaries	

\bibliographystyle{plainnat}
\bibliography{references}

\end{document}
